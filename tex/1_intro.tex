\section{Introduction} \label{sec:intro}% 1-2 mins

% background + motivation for work
\subsection{Background} \label{ssec:intro_bg}
\begin{frame}{Background}

	\begin{figure}[t]
		\includegraphics<4-4>[width=\textwidth,height=0.8\textheight,keepaspectratio]{acute-cystic-fibrosis}
	\end{figure}

  \begin{itemize}
  \item<1-3,5| alert@+> 200,000 hospital acquired infections (HAIs) occur annually in Australia.
  \item<2-3,5| alert@+> In-patient care vs out-patient care
  \item<3-3,5| alert@+> Cystic Fibrosis (CF) is a genetic condition that primarily affects the lungs.
  \item<5-| alert@5> CF health care delivery has moved to out-patient environments.
  \end{itemize}
  
\end{frame}

\subsection{Research Outline} \label{ssec:intro_research}
\begin{frame}{Research Outline}

	\begin{block}<2->{Hypothesis}
    	Our hypothesis is that patient encounters can be tracked using lightweight indoor localisation technologies allowing for interventions to improve patient flow, reduce patient contact, and reduce HAIs.
    \end{block}
    
    \begin{block}<3->{Aim}
    	Identify areas of potential cross infection in the hospital out-patient environment.
    \end{block}
    
\end{frame}

\subsection{Scope} \label{ssec:intro_scope}
\begin{frame}{Scope}

	\begin{itemize}[<+-| alert@+>]
      \item Android smart-phone
      	\begin{itemize}
			\item presence of low cost embedded sensors in smart-phones
            \item ubiquity and support available for android development
            \item ease and simplicity involved in implementation
	   	\end{itemize}
      \item Air-borne infection transmission among CF patients
      \item SNA focused on disease transmission and control
  	\end{itemize}

\end{frame}

\subsection{Objectives} \label{ssec:intro_obj}
\begin{frame}{Objectives}
	
	\begin{itemize}[<+-| alert@+>]
		\item Investigation into an accurate and scalable indoor RTLS approach for tracking patient movements.
		\item Development of a smart-phone application to accurately track the position of the CF patient indoors.
		\item Development of algorithms to identify high risk areas for CF patients in the hospital out-patient environment.
		\item Implementation and testing of the software system to identify areas of improvement and practicality of system.
	\end{itemize}
	
\end{frame}
