\section{Results} \label{sec:results}% 2-3 mins

\subsection{Step Detection} \label{ssec:results_stepD}
\begin{frame}{Step Detection}
	
	\begin{figure}
		\centering
		\includegraphics[scale = 0.4]{stepD}
	\end{figure}

\end{frame}

\subsection{Heading Estimation} \label{ssec:results_headinEst}
\begin{frame}{Heading Estimation}
	
		\begin{figure}
			\centering
			\includegraphics[scale = 0.4]{headingEst}
		\end{figure}

\end{frame}

\subsection{Map Matching} \label{ssec:results_mapM}
\begin{frame}{Map Matching}
	
	\begin{figure}
		\centering
		\includegraphics[scale = 0.4]{mapM}
	\end{figure}
	
\end{frame}